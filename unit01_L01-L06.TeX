\documentclass[11pt]{article}

% basic packages
\usepackage[margin=1in]{geometry}
\usepackage[pdftex]{graphicx}
\usepackage{amsmath,amssymb,amsthm}
\usepackage{custom}
\usepackage{lipsum}
\usepackage{enumitem}
\usepackage{hyperref}

% page formatting
\usepackage{fancyhdr}
\pagestyle{fancy}

\renewcommand{\sectionmark}[1]{\markright{\textsf{\arabic{section}. #1}}}
\renewcommand{\subsectionmark}[1]{}
\lhead{\scriptsize\textbf{\thepage}\ \ \nouppercase{\rightmark}}
\chead{}
\rhead{\scriptsize MAP2302 - Spring 2026}
\lfoot{}
\cfoot{}
\rfoot{}
\setlength{\headheight}{14pt}

\linespread{1.03} % give a little extra room
\setlength{\parindent}{0pt}
\setlength{\parskip}{6pt}
\setcounter{secnumdepth}{2} % no numbered subsubsections
\setcounter{tocdepth}{2} % no subsubsections in ToC

\begin{document}

% make title page
\thispagestyle{empty}
\bigskip
\vspace{0.1cm}

\begin{center}
{\fontsize{22}{22}\selectfont Lecture Notes on}
\vskip 16pt
{\fontsize{25}{25}\selectfont \bf \sffamily Unit 01 Differential Equations}
\vskip 24pt
{\fontsize{18}{18}\selectfont \rmfamily Aspen J. Johnson}
\vskip 6pt
{\fontsize{14}{14}\selectfont \ttfamily aspen.johnson @ my.pbsc.edu}
\vskip 24pt
\end{center}

{\parindent0pt \baselineskip=15.5pt}

The contents of these pages constitute my authorized exam notes for Unit 01 of Differential Equations. This exam will take place at the
Lake Worth campus under the supervision of Professor Tamara Johns on Wednesday, February 18, 2026, at 10:00 AM, in her physical office.

This is a retake opportunity granted after my initial Exam 01 Attempt 01 score of 25\%, and these notes have been prepared to support a stronger
performance on the retake. During the exam, I am permitted to reference these notes. The material included here is compiled directly from the following
resources, which I consulted while preparing:

\begin{enumerate}
    \item Lecture recordings and handouts from Prof.\ Johns
    \item \emph{Differential Equations} textbook by Blanchard, Hall, and Devaney
    \item \href{https://www.jirka.org/diffyqs/html/fo_chapter.html}{Online notes by Professor Lebl}
    \item \href{https://youtube.com/playlist?list=PLbYOkEbzW902GJZswbMnWHQaP4kKMS2TV&si=WetwPhjknIVybtdj}{Houston Math's YouTube channel}
\end{enumerate}

% make table of contents
\newpage
\microtoc
\newpage


%%%%%%%%%%%%%%%%%%%%%%%%%%%%%%%%%
% MAIN CONTENT %
%%%%%%%%%%%%%%%%%%%%%%%%%%%%%%%%%

% Lesson 01 - DEFINITIONS AND TERMINOLOGY, IVP, AND SLOPE FIELDS
\section{DEFINITIONS AND TERMINOLOGY, IVP, AND SLOPE FIELDS}
\subsection{Definitions and Terminology}

\subsection{Initial Value Problems}
With initial value problems, we are given a differential equation and then given a point in the form of $y(x)=y_0$ to be substituted into the general
solution (once found) and then used to solve for the arbitrary constant(s).

\textbf{Example Problems and Solutions From Houston Math}

\begin{enumerate}
  \item Answer the following parts corresponding to each differential equation below:
  \begin{align}
    \frac{dy}{dx} &= 2x \\
    y^{\prime} &= 6x^{2} + 4 \\
    x^{2}y^{\prime} &= -1 \\
    y^{\prime\prime} &= xe^{x}
  \end{align}

  \begin{enumerate}[topsep=2pt, itemsep=2pt, leftmargin=*]
    \item Find the \textbf{general solution}.
    \item Find the \textbf{particular solution} given that $y(1)=4$.
    \item Find the \textbf{particular solution} given that $y(1)=3$.
  \end{enumerate}

  \item Answer the following parts corresponding to the \textbf{second-order differential equation}:
  \begin{align}
    \frac{d^{2}y}{dt^{2}} = \cos(t)
  \end{align}

  \begin{enumerate}[topsep=2pt, itemsep=2pt, leftmargin=*]
    \item Find the \textbf{particular solution} given that $y^{\prime}(0) = 0$ and $y(0) = 1$.
    \item Find the \textbf{particular solution} given that $y(1)=3$.
  \end{enumerate}
\end{enumerate}

\newpage
% Lesson 02 - SEPARABLE EQUATIONS
\section{SEPARABLE EQUATIONS}
\subsection{(2.2) - (General Overview) Separable Equations}
If a \textbf{differential equation} can be written with all of the dependent-variable
expressions on one side (usually $y$), and all of the independent-variable expressions
on the other side (usually $x$), then we say that the equation is \textbf{separable}.

The form we desire when identifying a \textbf{separable equation} is $\frac{dy}{g(y)} = f(x)\,d(x)$. If
the form is not easy to spot at first glance, then this is the form that we wish to rewrite the given differential equation into.

It is also important that both sides are integrable. These antiderivatives may
differ by a constant at most, but what we really care about is that the right-hand side of the equation is integrable.

Small examples where the equations are already in the form \textbf{$h(y)\,dy = g(x)\,dx$}:

\begin{align}
y^{2}\,dy &= 4x\,dx \\
\frac{dy}{y} &= t e^{t}\,dt \\
\sec(t)\tan(t)\,dt &= dx \\
\frac{x+1}{x-1}\,dx &= \frac{dy}{y^{2}+1}
\end{align}

Here is an example where the equation is not in the form, but is separable (with a bit of work):
\begin{align}
\frac{dy}{dx} - x = xy^{2}
\end{align}

Here is an example of an equation that is \textbf{not} separable:
\begin{align}
\frac{dy}{dx} - x = y^{2}
\end{align}
The above \textbf{is not separable} because there is no way to create a product between $x$'s and $y$'s such that
we attain the form \textbf{$h(y)\,dy = g(x)\,dx$} or \textbf{$\frac{dy}{g(y)} = f(x)\,d(x)$}.

\noindent\rule{\linewidth}{0.4pt}
\textbf{General Outline For Solving A Separable Equation:}
\begin{enumerate}[topsep=2pt, itemsep=2pt, leftmargin=*]
    \item Separate the variables, making sure each differential is on top.
    \item Integrate both sides \textbf{with respect to their particular variable}. For example:
    \begin{enumerate}
        \item In $\frac{dy}{dx}$, $y$ is the dependent variable, as it is \textbf{dependent} on $x$.
        \item In $\frac{dy}{dt}$, $y$ is the dependent variable, as it is \textbf{dependent} on $t$.
        \item In $\frac{dx}{dt}$, $x$ is the dependent variable, as it is \textbf{dependent} on $t$.
    \end{enumerate}
    \item Solve for the dependent variable, if reasonable.
\end{enumerate}
\noindent\rule{\linewidth}{0.4pt}
\newpage

\textbf{Example Problems On Separable Equations From Houston Math}

\begin{enumerate}
  \item Solve the following separable differential equations:
  \begin{align}
    {y^{\prime}} &= \frac{x}{y} \\
    \frac{dy}{dx} - x &= xy^{2}\\
    x^{2}y^{\prime} &= -1 \\
    y^{\prime\prime} &= xe^{x}
  \end{align}

  \begin{enumerate}[topsep=2pt, itemsep=2pt, leftmargin=*]
    \item Find the \textbf{general solution}.
    \item Find the \textbf{particular solution} given that $y(1)=4$.
    \item Find the \textbf{particular solution} given that $y(1)=3$.
  \end{enumerate}

  \item Answer the following parts corresponding to the \textbf{second-order differential equation}:
  \begin{align}
    \frac{d^{2}y}{dt^{2}} = \cos(t)
  \end{align}

  \begin{enumerate}[topsep=2pt, itemsep=2pt, leftmargin=*]
    \item Find the \textbf{particular solution} given that $y^{\prime}(0) = 0$ and $y(0) = 1$.
    \item Find the \textbf{particular solution} given that $y(1)=3$.
  \end{enumerate}
\end{enumerate}

\subsection{Exponential Change}

The point of this section is to show how exponential change can be written and interpreted as
a differential equation, specifically a \textbf{separable differential equation}.

\begin{align}
  \frac{dy}{dt} = ky
\end{align}

As shown above, the equation represents the rate of change of some quantity $\left(\frac{dy}{dt}\right)$
being equal to some multiple of itself ($k$), which is exponential change.

\newpage
\textbf{We can solve the exponential separable equation as follows:}

\begin{align*}
\frac{dy}{dx} &= xy \\
\frac{1}{y}\,dy &= x\,dx \\
\int \frac{1}{y}\,dy &= \int x\,dx \\
\ln|y| &= \frac{x^2}{2} + C \\
y &= Ce^{x^2/2}
\end{align*}

\begin{itemize}
  \item $y$ = final amount
  \item $C$ = constant determined by the initial condition
  \item When $k > 0$, we have \textbf{growth}
  \item When $k < 0$, we have \textbf{decay}
\end{itemize}

\subsection{Initial Value Problems (continued)}
\textbf{Problems On Initial Value Problems}

\begin{enumerate}
  \item Solve the following \textbf{separable equations}:
  \begin{align}
    \frac{dy}{dx} &= 3x^{2} \text{, where y(0) = 2}\\
    \frac{dy}{dx}  &= xy \text{, where y(1) = 4}\\
    y^{\prime} &= \frac{y}{1+x^{2}} \text{, where y(0) = 5}\\
    y^{\prime\prime} &= xe^{x} \text{, where y(0) = 0} \\
    y^{\prime\prime} &= xe^{x} \text{, where y(2) = 3}
  \end{align}

  \item Solve the following \textbf{first-order linear equations}:
  \begin{align}
    y^{\prime} + 2y &= e^{-x} \text{, where } y(0) = 1\\
    y^{\prime} - \frac{1}{x}y &= x^{2} \text{, where } y(1) = 0\\
    y^{\prime} + (\tan x)\,y &= \sec x \text{, where } y(0) = 2\\
    y^{\prime} + \frac{2}{1+x}y &= (1+x)^{2} \text{, where } y(0) = 1
  \end{align}
\end{enumerate}

\newpage
% Lesson 03 - LINEAR EQUATIONS
\section{LINEAR EQUATIONS}
\subsection{(2.3) - (General Overview) Linear Equations}
A \textbf{first-order linear differential equation} is one that can be written in the standard form $y' + P(x)y = q(x)$, where $P(x)$ and $q(x)$ are known functions of $x$. 

These equations are important because they have a consistent solution process: we compute an integrating factor to rewrite the left-hand side as the derivative of a product, 
integrate both sides, and then apply an initial condition (if given) to find the particular solution.

\subsection{Integrating Factor Method}

The integrating factor method is a reliable technique for solving \textbf{first-order linear} differential equations. Its importance is
that it takes an equation where the unknown function ($y$) and its derivative ($y'$) are mixed together and turns it into a form that can be
integrated directly.

We do this by multiplying the entire equation by a carefully chosen function (the \textbf{integrating factor}) so that the left-hand
side becomes the derivative of a single product.

Once that happens, the problem becomes straightforward: integrate both sides and then use the initial condition to determine the constant and
produce the specific solution.

\noindent\textbf{First-Order Linear Equations (Integrating Factor Form).}
A first-order linear differential equation can be written in the standard form
\[
y' + P(x)\,y = q(x).
\]
In this form, $P(x)$ and $q(x)$ are known functions of $x$, and the integrating factor method applies directly.

\medskip
\noindent\textbf{Examples (identify $P(x)$ and $q(x)$):}

\begin{align*}
y' + 3y &= 6
\end{align*}
\begin{itemize}
  \item $P(x)=3$
  \item $q(x)=6$
\end{itemize}

\begin{align*}
y' + \frac{1}{x}y &= e^{x}
\end{align*}
\begin{itemize}
  \item $P(x)=\frac{1}{x}$
  \item $q(x)=e^{x}$
\end{itemize}

\begin{align*}
xy' + 3y &= x
\end{align*}
\noindent Dividing both sides by $x$ (assuming $x\neq 0$) gives the standard form:
\begin{align*}
y' + \frac{3}{x}y &= 1
\end{align*}
\begin{itemize}
  \item $P(x)=\frac{3}{x}$
  \item $q(x)=1$
\end{itemize}

\newpage

\noindent\textbf{Problem:} Solve the linear differential equation
\[
x^{2}y' + 5xy = x.
\]

\noindent\textbf{Step 1: Put in standard linear form.} Divide both sides by $x^{2}$ (assuming $x\neq 0$):
\[
y' + \frac{5}{x}y = \frac{1}{x}.
\]

\noindent\textbf{Step 2: Compute the integrating factor.}
\begin{align*}
\mu(x) &= e^{\int \frac{5}{x}\,dx}
      = e^{5\ln|x|}.
\end{align*}

\noindent\textbf{Inverse property (log/exponential).} Since $e^{\ln(a)}=a$ for $a>0$,
\begin{align*}
e^{5\ln|x|}
&= \left(e^{\ln|x|}\right)^{5}
= |x|^{5}.
\end{align*}
On an interval where $x>0$, this simplifies to $\mu(x)=x^{5}$.

\noindent\textbf{Step 3: Multiply the differential equation by $\mu(x)$.} (Using $x>0$ so $\mu(x)=x^{5}$.)
\begin{align*}
x^{5}y' + x^{5}\left(\frac{5}{x}\right)y &= x^{5}\left(\frac{1}{x}\right) \\
x^{5}y' + 5x^{4}y &= x^{4}.
\end{align*}

\noindent\textbf{Step 4: Recognize the product rule form.}
\[
\frac{d}{dx}\left(x^{5}y\right)=x^{5}y' + 5x^{4}y,
\]
so the equation becomes
\[
\frac{d}{dx}\left(x^{5}y\right)=x^{4}.
\]

\noindent\textbf{Step 5: Integrate both sides and solve for $y$.}
\begin{align*}
\int \frac{d}{dx}\left(x^{5}y\right)\,dx &= \int x^{4}\,dx \\
x^{5}y &= \frac{x^{5}}{5} + C \\
y &= \frac{1}{5} + \frac{C}{x^{5}}.
\end{align*}

\noindent\textbf{Final solution:}
\[
\boxed{\;y = \frac{1}{5} + \frac{C}{x^{5}}.\;}
\]

\newpage
%========================================================
% Exact Equations — Two Worked Examples + General Outline
%========================================================

\section{Exact Equations}

\noindent An equation of the form $M(x,y)\,dx + N(x,y)\,dy = 0$ is \textbf{exact} if there exists a function $\psi(x,y)$ such that
$\psi_x = M$ and $\psi_y = N$.
Equivalently, the exactness test is
\[
\frac{\partial M}{\partial y}=\frac{\partial N}{\partial x}.
\]

%-----------------------------
% Example 01
%-----------------------------
\subsection{Example 1}

\noindent\textbf{Solve:} $(x^2+xy^2)\,dx + (yx^2-y^3)\,dy = 0$

\medskip
\noindent\textbf{Step 1: Identify $M$ and $N$.}
\[
M(x,y)=x^2+xy^2,
\qquad
N(x,y)=yx^2-y^3.
\]

\noindent\textbf{Step 2: Test for exactness.}
\begin{align*}
\frac{\partial M}{\partial y} &= \frac{\partial}{\partial y}\left(x^2+xy^2\right)=2xy,\\
\frac{\partial N}{\partial x} &= \frac{\partial}{\partial x}\left(yx^2-y^3\right)=2xy.
\end{align*}
\noindent Since $\frac{\partial M}{\partial y}=\frac{\partial N}{\partial x}$, the equation is \textbf{exact}.

\medskip
\noindent\textbf{Step 3: Find a potential function $\psi(x,y)$.}
Integrate $M$ with respect to $x$:
\begin{align*}
\psi(x,y) &= \int \left(x^2+xy^2\right)\,dx \\
          &= \frac{x^3}{3}+\frac{x^2}{2}y^2 + h(y)
          \qquad\text{($h(y)$ is constant with respect to $x$).}
\end{align*}

\noindent\textbf{Step 4: Differentiate $\psi$ with respect to $y$ and match $N$.}
\begin{align*}
\psi_y(x,y) &= \frac{\partial}{\partial y}\left(\frac{x^3}{3}+\frac{x^2}{2}y^2 + h(y)\right) \\
            &= x^2y + h'(y).
\end{align*}
Set $\psi_y = N$:
\[
x^2y+h'(y)=yx^2-y^3
\quad\Longrightarrow\quad
h'(y)=-y^3
\quad\Longrightarrow\quad
h(y)=-\frac{y^4}{4}.
\]

\noindent\textbf{Final implicit solution:}
\[
\boxed{\;\frac{x^3}{3}+\frac{x^2y^2}{2}-\frac{y^4}{4}=C\;}
\]

%-----------------------------
% Example 02
%-----------------------------
\subsection{Example 2}

\noindent\textbf{Solve:} $\displaystyle \frac{dy}{dx}=\frac{3ye^x-4xe^y}{2x^2e^y-3e^x}$

\medskip
\noindent\textbf{Step 1: Rewrite in the form $M\,dx+N\,dy=0$.}
\begin{align*}
\frac{dy}{dx} &= \frac{3ye^x-4xe^y}{2x^2e^y-3e^x} \\
\left(2x^2e^y-3e^x\right)\,dy &= \left(3ye^x-4xe^y\right)\,dx \\
\left(4xe^y-3ye^x\right)\,dx + \left(2x^2e^y-3e^x\right)\,dy &= 0.
\end{align*}

\noindent\textbf{Step 2: Identify $M$ and $N$.}
\[
M(x,y)=4xe^y-3ye^x,
\qquad
N(x,y)=2x^2e^y-3e^x.
\]

\noindent\textbf{Step 3: Test for exactness.}
\begin{align*}
\frac{\partial M}{\partial y} &= \frac{\partial}{\partial y}\left(4xe^y-3ye^x\right)=4xe^y-3e^x,\\
\frac{\partial N}{\partial x} &= \frac{\partial}{\partial x}\left(2x^2e^y-3e^x\right)=4xe^y-3e^x.
\end{align*}
\noindent Since $\frac{\partial M}{\partial y}=\frac{\partial N}{\partial x}$, the equation is \textbf{exact}.

\medskip
\noindent\textbf{Step 4: Find a potential function $\psi(x,y)$.}
Integrate $N$ with respect to $y$:
\begin{align*}
\psi(x,y) &= \int \left(2x^2e^y-3e^x\right)\,dy \\
          &= 2x^2e^y - 3ye^x + g(x)
          \qquad\text{($g(x)$ is constant with respect to $y$).}
\end{align*}

\noindent\textbf{Step 5: Differentiate $\psi$ with respect to $x$ and match $M$.}
\begin{align*}
\psi_x(x,y) &= \frac{\partial}{\partial x}\left(2x^2e^y - 3ye^x + g(x)\right) \\
            &= 4xe^y - 3ye^x + g'(x).
\end{align*}
Set $\psi_x = M$:
\[
4xe^y - 3ye^x + g'(x) = 4xe^y - 3ye^x
\quad\Longrightarrow\quad
g'(x)=0,
\]
so $g(x)$ is constant and can be absorbed into $C$.

\noindent\textbf{Final implicit solution:}
\[
\boxed{\;2x^2e^y - 3ye^x = C\;}
\]

\subsection{General Outline for Solving an Exact Equation}

\noindent\rule{\linewidth}{0.4pt}
\textbf{General Outline For Solving An Exact Equation:}
\begin{enumerate}[topsep=2pt, itemsep=2pt, leftmargin=*]
    \item Rewrite the equation in the form $M(x,y)\,dx + N(x,y)\,dy = 0$.
    \item Compute $\frac{\partial M}{\partial y}$ and $\frac{\partial N}{\partial x}$.
    \item If $\frac{\partial M}{\partial y}=\frac{\partial N}{\partial x}$, then the equation is \textbf{exact}.
    \item Find a potential function $\psi(x,y)$:
    \begin{enumerate}[topsep=2pt, itemsep=2pt, leftmargin=*]
        \item Integrate $M(x,y)$ with respect to $x$ to get $\psi(x,y)=\int M\,dx + h(y)$, \textbf{or}
        \item Integrate $N(x,y)$ with respect to $y$ to get $\psi(x,y)=\int N\,dy + g(x)$.
    \end{enumerate}
    \item Differentiate $\psi$ with respect to the other variable and match it to the remaining function
    ($\psi_y=N$ or $\psi_x=M$) to determine $h(y)$ or $g(x)$.
    \item Write the implicit solution as $\psi(x,y)=C$.
\end{enumerate}
\noindent\rule{\linewidth}{0.4pt}
\newpage

% Lesson 05 - SOLUTIONS BY SUBSTITUTION
\section{SOLUTIONS BY SUBSTITUTION}
\subsection{(2.5) - (General Overview) Solutions By Substitutions}
Solutions by substitution are used when a differential equation does not appear separable or linear at first, but can be transformed into a familiar form by introducing a new variable.

The main idea is to recognize a pattern in the equation and choose a substitution that simplifies the algebra and reduces the problem to a standard method you already know.

In this section, we will focus on \textbf{common substitution-based techniques}, including homogeneous equations (often handled by substituting $v = y/x$), \textbf{Bernoulli equations} (which become linear after an appropriate substitution),
and numerical approaches such as \textbf{Euler’s method}, which approximates solutions when an exact formula is difficult or impossible to obtain.

\subsection{Homogeneous Equations}
If the equation is in the form
\begin{align*}
  \frac{dy}{dx} = f(x,y),
\end{align*}
and it is true that
\begin{align*}
  f(tx,ty) = f(x,y),
\end{align*}
then the equation is homogeneous and we can deploy the substitution
\begin{align*}
  y = vx.
\end{align*}

If we differentiate $y=vx$, we obtain an expression that eliminates $y$:
\begin{align*}
  dy = v\,dx + x\,dv.
\end{align*}

\subsection{Bernoulli Equations}
A Bernoulli equation is a first-order differential equation that looks almost linear, except for a power of the unknown function.

The key idea is that even though the presence of $y^n$ (with $n\neq 0,1$) prevents it from being linear in $y$, it can be converted into a linear equation by an appropriate substitution.

After rewriting the equation in the standard Bernoulli form $y' + P(x)y = Q(x)y^n$, we substitute $v = y^{1-n}$. This choice is important because it transforms the nonlinear $y^n$ term into something involving $v$ and $v'$, producing a first-order linear equation in $v$.

Once the equation is linear, we solve it using the integrating factor method and then substitute back to recover $y$.

\noindent\textbf{Bernoulli Equation Example.} Solve
\[
\frac{dy}{dx}-\frac{1}{x}y = x y^{2}.
\]

\noindent\textbf{Step 1: Identify Bernoulli form and choose substitution.}
\[
y' + P(x)y = Q(x)y^n
\quad\Rightarrow\quad
P(x)=-\frac{1}{x},\; Q(x)=x,\; n=2.
\]
Let
\[
v = y^{1-n} = y^{-1}=\frac{1}{y}.
\]
Then
\[
\frac{dv}{dx} = -\frac{y'}{y^{2}}.
\]

\noindent\textbf{Step 2: Divide the DE by $y^{2}$ and rewrite in terms of $v$.}
\begin{align*}
\frac{dy}{dx}-\frac{1}{x}y &= x y^{2} \\
\frac{1}{y^{2}}\frac{dy}{dx}-\frac{1}{x}\frac{y}{y^{2}} &= x \\
\frac{y'}{y^{2}}-\frac{1}{x}\frac{1}{y} &= x \\
-\frac{dv}{dx}-\frac{1}{x}v &= x \qquad\text{(since $\frac{y'}{y^2}=-v'$ and $\frac{1}{y}=v$)}\\
\frac{dv}{dx}+\frac{1}{x}v &= -x \qquad\text{(multiply both sides by $-1$)}
\end{align*}

\noindent\textbf{Step 3: Integrating factor.}
\[
\mu(x)=e^{\int \frac{1}{x}\,dx}=e^{\ln|x|}=|x|.
\]
On an interval where $x>0$, take $\mu(x)=x$.

\noindent\textbf{Step 4: Multiply by $\mu(x)=x$ and use the product rule.}
\begin{align*}
x\left(v'+\frac{1}{x}v\right) &= x(-x) \\
xv' + v &= -x^{2} \\
\frac{d}{dx}(xv) &= -x^{2} \qquad\text{(because $(xv)'=xv' + v$)}
\end{align*}

\noindent\textbf{Step 5: Integrate and solve for $v$.}
\begin{align*}
\int \frac{d}{dx}(xv)\,dx &= \int -x^{2}\,dx \\
xv &= -\frac{x^{3}}{3}+C \\
v &= -\frac{x^{2}}{3}+\frac{C}{x}
\end{align*}

\noindent\textbf{Step 6: Substitute back $v=\frac{1}{y}$ and solve for $y$.}
\begin{align*}
\frac{1}{y} &= -\frac{x^{2}}{3}+\frac{C}{x} \\
y &= \frac{1}{\frac{C}{x}-\frac{x^{2}}{3}}
= \frac{1}{\frac{3C-x^{3}}{3x}}
= \frac{3x}{3C-x^{3}}.
\end{align*}

\noindent\textbf{Final solution:}
\[
\boxed{\;y = \frac{3x}{3C-x^{3}}.\;}
\]

\subsection{Homogeneous First-Order Differential Equation}

\noindent\textbf{Problem.} Solve the differential equation
\[
\frac{dy}{dx}=\frac{x}{y}+\frac{y}{x}.
\]

\noindent\textbf{Step 1: Test for homogeneity.}
Let
\[
f(x,y)=\frac{x}{y}+\frac{y}{x}.
\]
Check whether $f(tx,ty)=f(x,y)$:
\begin{align*}
f(tx,ty)
&= \frac{tx}{ty}+\frac{ty}{tx} \\
&= \frac{x}{y}+\frac{y}{x} \qquad\text{(the $t$'s cancel)}\\
&= f(x,y).
\end{align*}
\noindent Since $f(tx,ty)=f(x,y)$, the equation is \textbf{homogeneous} (degree $0$), so we use the substitution $y=vx$.

\medskip
\noindent\textbf{Step 2: Substitute $y=vx$ and rewrite $dy/dx$.}
\begin{align*}
y &= vx \qquad\text{(set $v=\frac{y}{x}$)}\\
\frac{dy}{dx} &= v + x\frac{dv}{dx} \qquad\text{(differentiate $y=vx$ using the product rule)}
\end{align*}

\medskip
\noindent\textbf{Step 3: Plug into the differential equation and separate variables.}
\begin{align*}
v + x\frac{dv}{dx} &= \frac{x}{vx} + \frac{vx}{x}
\qquad\text{(substitute $y=vx$ into the RHS)}\\
v + x\frac{dv}{dx} &= \frac{1}{v} + v \qquad\text{(simplify)}\\
x\frac{dv}{dx} &= \frac{1}{v} \qquad\text{(subtract $v$ from both sides)}\\
v\,dv &= \frac{1}{x}\,dx \qquad\text{(separate variables)}
\end{align*}

\medskip
\noindent\textbf{Step 4: Integrate.}
\begin{align*}
\int v\,dv &= \int \frac{1}{x}\,dx \\
\frac{1}{2}v^{2} &= \ln|x| + C \\
v^{2} &= 2\ln|x| + C \qquad\text{(multiply both sides by $2$)}
\end{align*}

\medskip
\noindent\textbf{Step 5: Resubstitute $v=\dfrac{y}{x}$ and solve for $y$.}
\noindent We solve for $v$ in terms of $x$ because $v=\dfrac{y}{x}$ is the substitution that connects back to the original variables.
\begin{align*}
\left(\frac{y}{x}\right)^{2} &= 2\ln|x| + C \qquad\text{(since $v=\frac{y}{x}$)}\\
y^{2} &= x^{2}\big(2\ln|x| + C\big) \qquad\text{(multiply both sides by $x^{2}$)}\\
y &= \pm x\sqrt{\,2\ln|x| + C\,} \qquad\text{(take square roots)}
\end{align*}

\noindent\textbf{Final solution:}
\[
\boxed{\;y = \pm x\sqrt{\,2\ln|x| + C\,}.\;}
\]


%========================================================
% Lesson 06 - LINEAR MODELS (Expanded + Worked Examples)
% General outlines placed at the end of their sections
%========================================================

\section{LINEAR MODELS}
\subsection{(3.1) - (General Overview) Modeling with 1st-Order Differential Equations}

Many real-world models begin by identifying a changing quantity and writing
\[
\frac{d(\text{amount})}{dt}=\text{(rate in)}-\text{(rate out)}.
\]
After the model is written, we solve the resulting differential equation (often separable or linear) and use initial conditions to determine constants.

\subsection{Exponential Growth and Decay}

\noindent \textbf{Proportional Change (Growth/Decay).}
If the rate of change of a quantity is proportional to the quantity itself, then
\[
\frac{dP}{dt}=kP,
\]
where $k>0$ represents growth and $k<0$ represents decay.

\subsubsection*{Example: Population Growth (Exponential)}
Suppose a population $P(t)$ grows at a rate proportional to its size:
\[
\frac{dP}{dt}=kP,\qquad P(0)=P_0.
\]
\begin{align*}
\frac{1}{P}\,dP &= k\,dt\\
\int \frac{1}{P}\,dP &= \int k\,dt\\
\ln|P| &= kt + C\\
P &= Ce^{kt}.
\end{align*}
Using $P(0)=P_0$ gives $P_0=C$, so
\[
\boxed{\,P(t)=P_0e^{kt}\,}.
\]

\subsubsection*{Radioactive Decay}
Radioactive decay is modeled the same way, but with $k<0$:
\[
\frac{dN}{dt}=kN,\qquad k<0.
\]
Thus the amount remaining is
\[
\boxed{\,N(t)=N_0e^{kt}\,}.
\]

%-----------------------------
% Outline (Growth/Decay)
%-----------------------------
\noindent\rule{\linewidth}{0.4pt}
\textbf{General Outline For Exponential Growth and Decay:}
\begin{enumerate}[topsep=2pt, itemsep=2pt, leftmargin=*]
    \item Identify the changing quantity and name it $Y(t)$.
    \item If the rate is proportional to the amount present, write
    \[
    \frac{dY}{dt}=kY,
    \]
    where $k>0$ implies growth and $k<0$ implies decay.
    \item Separate variables and integrate:
    \[
    \frac{1}{Y}\,dY = k\,dt
    \quad\Rightarrow\quad
    \ln|Y|=kt+C.
    \]
    \item Solve for $Y(t)$:
    \[
    Y(t)=Ce^{kt}.
    \]
    \item Use the initial condition $Y(0)=Y_0$ to get $C=Y_0$ and write
    \[
    \boxed{\,Y(t)=Y_0e^{kt}\,}.
    \]
\end{enumerate}
\noindent\rule{\linewidth}{0.4pt}

\subsection{Logistic Growth}

With logistic growth, an example is population growth. A key feature of logistic growth is that there is a ``limit''
or cap on how far we can grow.

\begin{center}
  $\frac{dP}{dt} = r\left(\frac{k - P}{k}\right)$
\end{center}

\begin{itemize}
  \item $\frac{dP}{dt}$ is the change in population
  \item $r$ is the rate
  \item $P$ is the current population
  \item $\frac{k - P}{k}$ is the carrying capacity ($k$), or the ``limit'' that we cannot grow beyond
\end{itemize}

\textbf{How We Write Population Growth Depends On:}
\begin{enumerate}
  \item A rate
  \item The size of the current population
  \item The growth slowing as the population approaches a ``carrying capacity''
\end{enumerate}

\textbf{After working this as a first-order separable equation, we get:}
\begin{center}
  \Large$P = \frac{k}{1 + ce^{-rt}}$
\end{center}

\subsubsection*{Example: Logistic Growth With an Initial Condition}
If
\[
P(t)=\frac{k}{1+ce^{-rt}}
\quad\text{and}\quad
P(0)=P_0,
\]
then
\begin{align*}
P_0 &= \frac{k}{1+c}\\
1+c &= \frac{k}{P_0}\\
c &= \frac{k-P_0}{P_0}.
\end{align*}
So
\[
\boxed{\,P(t)=\frac{k}{1+\left(\frac{k-P_0}{P_0}\right)e^{-rt}}\,}.
\]

%-----------------------------
% Outline (Logistic)
%-----------------------------
\noindent\rule{\linewidth}{0.4pt}
\textbf{General Outline For Logistic Growth:}
\begin{enumerate}[topsep=2pt, itemsep=2pt, leftmargin=*]
    \item Let $P(t)$ be the population and identify the carrying capacity $k$.
    \item Write the logistic model:
    \[
    \frac{dP}{dt}=rP\left(1-\frac{P}{k}\right)
    \quad\text{(equivalently } \frac{dP}{dt}=rP\left(\frac{k-P}{k}\right)\text{)}.
    \]
    \item Separate variables and integrate (partial fractions are typically required).
    \item Solve for $P(t)$ and write the standard solution form:
    \[
    P(t)=\frac{k}{1+ce^{-rt}}.
    \]
    \item Use $P(0)=P_0$ to determine $c=\dfrac{k-P_0}{P_0}$.
\end{enumerate}
\noindent\rule{\linewidth}{0.4pt}

\newpage
\subsection{Newton's Law Of Cooling}

Newton's Law of Cooling states that the rate of change in the temperature of an object
is some multiple of the difference between its temperature and the temperature of the surrounding medium.

\begin{center}
  \large$\frac{dT}{dt} = k(T - T_{m})$\normalsize{, where}
\end{center}

\begin{enumerate}
  \item \textbf{$\frac{dT}{dt}$} is the rate of change of temperature, $T$, over time
  \item \textbf{$k$} is the constant multiple
  \item \textbf{$(T - T_{m})$} is the difference between:
   \begin{enumerate}
    \item the object ($T$)
    \item the surrounding medium ($T_{m}$)
   \end{enumerate}
\end{enumerate}

\textbf{After working this as a first-order separable equation, we get:}
\begin{center}
  \large$T = ce^{kt} + T_{m}$
\end{center}

\subsubsection*{Example: Cooling With Initial Temperature}
If $T(0)=T_0$, then
\begin{align*}
T(0) &= ce^{k\cdot 0}+T_m\\
T_0 &= c+T_m\\
c &= T_0-T_m.
\end{align*}
So
\[
\boxed{\,T(t)=T_m+(T_0-T_m)e^{kt}\,}.
\]

%-----------------------------
% Outline (Newton's Law of Cooling)
%-----------------------------
\noindent\rule{\linewidth}{0.4pt}
\textbf{General Outline For Newton's Law of Cooling (or Heating):}
\begin{enumerate}[topsep=2pt, itemsep=2pt, leftmargin=*]
    \item Let $T(t)$ be the temperature of the object and let $T_m$ be the (constant) surrounding temperature.
    \item Write the model:
    \[
    \frac{dT}{dt}=k\left(T-T_m\right),
    \]
    where typically $k<0$ for cooling (object temperature approaches $T_m$ over time).
    \item Separate variables:
    \[
    \frac{1}{T-T_m}\,dT = k\,dt.
    \]
    \item Integrate:
    \[
    \ln|T-T_m| = kt + C.
    \]
    \item Solve for $T(t)$:
    \[
    T-T_m = Ce^{kt}
    \quad\Rightarrow\quad
    T(t)=T_m+Ce^{kt}.
    \]
    \item Use the initial condition $T(0)=T_0$ to get $C=T_0-T_m$, so
    \[
    \boxed{\,T(t)=T_m+(T_0-T_m)e^{kt}\,}.
    \]
\end{enumerate}
\noindent\rule{\linewidth}{0.4pt}


\subsection{Mixing Problems}

\noindent \textbf{General Setup (Salt/Sugar Mixing).}
Let $A(t)$ be the amount of solute (e.g., salt) in the tank at time $t$.
A standard mixing model is
\[
\frac{dA}{dt}=\text{(rate in)}-\text{(rate out)}.
\]
Typically,
\[
\text{rate in}=(\text{inflow concentration})\cdot(\text{inflow rate}),
\qquad
\text{rate out}=(\text{tank concentration})\cdot(\text{outflow rate}),
\]
where $\text{tank concentration}=\dfrac{A(t)}{V(t)}$ and $V(t)$ is the volume.

\subsubsection*{Example: Constant Volume Mixing (Linear DE)}
A tank has constant volume $V$ gallons. Brine enters at rate $r$ (gal/min) with concentration $c_{\text{in}}$ (lb/gal) and leaves at the same rate $r$.
Then
\[
\frac{dA}{dt}=rc_{\text{in}}-r\frac{A}{V}.
\]
This is linear:
\[
\frac{dA}{dt}+\frac{r}{V}A=rc_{\text{in}}.
\]
Solving gives
\[
\boxed{\,A(t)=Vc_{\text{in}}+\left(A_0-Vc_{\text{in}}\right)e^{-(r/V)t}\,},
\]
where $A(0)=A_0$.

%-----------------------------
% Outline (Mixing)
%-----------------------------
\noindent\rule{\linewidth}{0.4pt}
\textbf{General Outline For Solving a Mixing Problem:}
\begin{enumerate}[topsep=2pt, itemsep=2pt, leftmargin=*]
    \item Let $A(t)$ be the amount of solute in the tank at time $t$.
    \item Write the model using
    \[
    \frac{dA}{dt}=\text{(rate in)}-\text{(rate out)}.
    \]
    \item Compute rate in:
    \[
    \text{rate in}=(\text{inflow concentration})\cdot(\text{inflow rate}).
    \]
    \item Compute rate out:
    \[
    \text{rate out}=\left(\frac{A(t)}{V(t)}\right)\cdot(\text{outflow rate}),
    \]
    where $V(t)$ is the volume in the tank.
    \item Determine $V(t)$ (constant volume if inflow = outflow; otherwise $V(t)$ changes with time).
    \item Solve the resulting differential equation (often linear in $A(t)$).
    \item Apply the initial condition $A(0)=A_0$.
\end{enumerate}
\noindent\rule{\linewidth}{0.4pt}

\subsection{Motion in One Direction}

\noindent \textbf{General Idea.}
In one dimension, common modeling relationships are:
\[
v=\frac{dx}{dt},
\qquad
a=\frac{dv}{dt}.
\]
If a force (or acceleration) depends on velocity, position, or time, we can create a differential equation.

\subsubsection*{Example 1: Constant Acceleration}
If $a(t)=a$ (constant), then
\[
\frac{dv}{dt}=a.
\]
Integrate:
\begin{align*}
v(t) &= at + C,\\
v(0)=v_0 \Rightarrow C&=v_0,
\end{align*}
so
\[
\boxed{\,v(t)=v_0+at\,}.
\]
Since $v=\dfrac{dx}{dt}$,
\[
\frac{dx}{dt}=v_0+at,
\]
and integrating again yields
\[
\boxed{\,x(t)=x_0+v_0t+\frac{1}{2}at^2\,}.
\]

\subsubsection*{Example 2: Linear Drag (Velocity-Dependent Resistance)}
A simple drag model assumes acceleration proportional to velocity:
\[
\frac{dv}{dt}=-kv,\qquad k>0.
\]
This is separable:
\begin{align*}
\frac{1}{v}\,dv &= -k\,dt\\
\ln|v| &= -kt + C\\
v &= Ce^{-kt}.
\end{align*}
If $v(0)=v_0$, then $C=v_0$, so
\[
\boxed{\,v(t)=v_0e^{-kt}\,}.
\]

%-----------------------------
% Outline (Motion)
%-----------------------------
\noindent\rule{\linewidth}{0.4pt}
\textbf{General Outline For Modeling Motion in One Dimension:}
\begin{enumerate}[topsep=2pt, itemsep=2pt, leftmargin=*]
    \item Choose variables:
    \begin{enumerate}[topsep=2pt, itemsep=2pt, leftmargin=*]
        \item $x(t)$ = position, $v(t)=\dfrac{dx}{dt}$ = velocity, $a(t)=\dfrac{dv}{dt}$ = acceleration.
    \end{enumerate}
    \item Use the given information to write a differential equation for either $v(t)$ or $x(t)$.
    \item Solve for $v(t)$ first if the model is written in terms of acceleration $a=\dfrac{dv}{dt}$.
    \item If position is needed, use $v=\dfrac{dx}{dt}$ and integrate to obtain $x(t)$.
    \item Apply initial conditions (such as $v(0)=v_0$ or $x(0)=x_0$) to determine constants.
\end{enumerate}
\noindent\rule{\linewidth}{0.4pt}


\subsection{Proportional Change Models}

\noindent Many models in this course reduce to the proportional change form
\[
\frac{dY}{dt}=kY
\quad\Rightarrow\quad
\boxed{\,Y(t)=Y_0e^{kt}\,}.
\]
Common applications include population growth (when $k>0$), radioactive decay (when $k<0$), interest models, and some simplified motion/resistance models.

\newpage


\end{document}
