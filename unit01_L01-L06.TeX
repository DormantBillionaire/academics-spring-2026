\documentclass[11pt]{article}

% basic packages
\usepackage[margin=1in]{geometry}
\usepackage[pdftex]{graphicx}
\usepackage{amsmath,amssymb,amsthm}
\usepackage{custom}
\usepackage{lipsum}
\usepackage{enumitem}
\usepackage{hyperref}

% page formatting
\usepackage{fancyhdr}
\pagestyle{fancy}

\renewcommand{\sectionmark}[1]{\markright{\textsf{\arabic{section}. #1}}}
\renewcommand{\subsectionmark}[1]{}
\lhead{\scriptsize\textbf{\thepage} \ \ \nouppercase{\rightmark}}
\chead{}
\rhead{\scriptsize MAP2302 - Spring 2026}
\lfoot{}
\cfoot{}
\rfoot{}
\setlength{\headheight}{14pt}

\linespread{1.03} % give a little extra room
\setlength{\parindent}{0pt}
\setlength{\parskip}{6pt}
\setcounter{secnumdepth}{2} % no numbered subsubsections
\setcounter{tocdepth}{2} % no subsubsections in ToC

\begin{document}

% make title page
\thispagestyle{empty}
\bigskip \
\vspace{0.1cm}

\begin{center}
{\fontsize{22}{22} \selectfont Lecture Notes on}
\vskip 16pt
{\fontsize{25}{25} \selectfont \bf \sffamily Unit 01 Differential Equations}
\vskip 24pt
{\fontsize{18}{18} \selectfont \rmfamily Aspen J. Johnson} 
\vskip 6pt
{\fontsize{14}{14} \selectfont \ttfamily aspen.johnson@my.pbsc.edu} 
\vskip 24pt
\end{center}

{\parindent0pt \baselineskip=15.5pt}

The contents of these pages constitute my authorized exam notes for Unit 01 of Differential Equations. This exam will take place at the 
Lake Worth campus under the supervision of Professor Tamara Johns on Wednesday, February 18, 2026 at 10:00 AM, in her physical office.

This is a retake opportunity granted after my initial Exam 01 Attempt 01 score of 25\%, and these notes have been prepared to support a stronger 
performance on the retake. During the exam, I am permitted to reference these notes. The material included here is compiled directly from the following 
resource(s), which I consulted while preparing:

\begin{enumerate}
    \item Lecture Recordings and Handouts from Prof. Johns
    \item Differential Equations Textbook By Blanchard, Hall, and Devaney
    \item \href{https://www.jirka.org/diffyqs/html/fo_chapter.html}{Online notes by Professor Lebl}
    \item \href{https://youtube.com/playlist?list=PLbYOkEbzW902GJZswbMnWHQaP4kKMS2TV&si=WetwPhjknIVybtdj}{Houston Math's Youtube Channel }
\end{enumerate}

% make table of contents
\newpage
\microtoc
\newpage


%%%%%%%%%%%%%%%%%%%%%%%%%%%%%%%%%
% MAIN CONTENT %
%%%%%%%%%%%%%%%%%%%%%%%%%%%%%%%%%

\newpage
% Lesson 01 - DEFINITIONS AND TERMINOLOGY, IVP, AND SLOPE FIELDS
\section{DEFINITIONS AND TERMINOLOGY, IVP, AND SLOPE FIELDS}
\subsection{Definitions and Terminology}

\subsection{Initial Value Problems }
With initial value problems, we are given a differential equation, and then given a point in the  form of y(x) = y to be plugged into the general solution (Once found) and then to be utilized to 
solve for the arbitrary constant(s). \\

\textbf{Example Problems and Solutions From Houston Math}

\begin{enumerate}
  \item Answer the following parts corresponding to each differential equation below:
  
  {
  \begin{align}
    \frac{dy}{dx} &= 2x \\
    y^{\prime} &= 6x^{2} + 4 \\
    x^{2}y^{\prime} &= -1 \\
    y^{\prime\prime} &= xe^{x}
  \end{align}
  }

  \begin{enumerate}[topsep=2pt, itemsep=2pt, leftmargin=*]
    \item Find the \textbf{general solution}.
    \item Find the \textbf{particular solution} given that $y(1)=4$.
    \item Find the \textbf{particular solution} given that $y(1)=3$.
  \end{enumerate}
  \item Answer the following parts corresponding to the \textbf{Second-Order Differential Equation}:
  {
  \begin{align}
    \frac{d^{2}y}{dt^{2}} = \cos(t)
  \end{align}
    }
  \begin{enumerate}[topsep=2pt, itemsep=2pt, leftmargin=*]
    \item Find the \textbf{particular solution} given that $y^{\prime}(0) = 0$ and $y(0) = 1$.
    \item Find the \textbf{particular solution} given that $y(1)=3$.
  \end{enumerate}
\end{enumerate}



\subsection{(Lesson 2.1) - Solution Curves Without A Formula }


\newpage
% Lesson 02 - SEPARABLE EQUATIONS 
\section{SEPARABLE EQUATIONS}
\subsection{(2.2) - (General Overview) Separable Equations}
If a \textbf{differential equation} can be written with all of the dependednt variable 
expressions on one side (usually $y$), and all of the independent  variable expressions 
on the other side(usually $x$), then we say that the equation is \textbf{seprarable}. 

The form we desire when identifying a \textbf{separable equation} is $\frac{dy}{g(y)} = f(x)\,d(x)$. If
the form is not easy to spot on first glance, then this is the form that we wish to get our given differential equation into. 

Something to note, is that we also want both sides to be integrable. These integrations or antiderivatives may 
differ by a constant at most, but what we really care about is that the right-side of the equation is integrable. 

Small examples where the equations are already of the form \textbf{$h(y)dy = g(x) dx$}

\begin{align}
y^{2}\,dy &= 4x\,dx \\
\frac{dy}{y} &= t e^{t}\,dt \\
\sec(t)\tan(t)\,dt &= dx \\
\frac{x+1}{x-1}\,dx &= \frac{dy}{y^{2}+1}
\end{align}

Here is an example where the equation is not in the form, but is separable (with a bit of work):
\begin{align}
\frac{dy}{dx} - x = xy^{2}
\end{align}

Here is an example of an equation that is \textbf{NOT} separable:
\begin{align}
\frac{dy}{dx} - x = y^{2}
\end{align}
The above \textbf{is not separable} because there is no way to create a product between x's and y's such that 
we attain the form \textbf{$h(y)dy = g(x) dx$} or \textbf{$\frac{dy}{g(y)} = f(x)\,d(x)$}.

\noindent\rule{\linewidth}{0.4pt}
\textbf{General Outline For Solving A Separable Equation}:
\begin{enumerate}[topsep=2pt, itemsep=2pt, leftmargin=*]
    \item Separate the variables, making sure each differential is on top 
    \item Integrate both sides \textbf{with respect to their particular variable}. For example:
    \begin{enumerate}
        \item In $\frac{dy}{dx}$, y is the dependent variable, as it is \textbf{dependent} on x.
        \item In $\frac{dy}{dt}$, y is the dependent variable, as it is \textbf{dependent} on t.
        \item In $\frac{dx}{dt}$, x is the dependent variable, as it is \textbf{dependent} on t.
    \end{enumerate}
    \item Solve for the dependent variable, if reasonable.
\end{enumerate}
\noindent\rule{\linewidth}{0.4pt}

\textbf{Example Problems On Separable Equations From Houston Math}

\begin{enumerate}
  \item Solve the following separable differential equations:
  
  {
  \begin{align}
    {y^{\prime}} &= \frac{x}{y} \\
    \frac{dy}{dx} - x &= xy^{2}\\
    x^{2}y^{\prime} &= -1 \\
    y^{\prime\prime} &= xe^{x}
  \end{align}
  }

  \begin{enumerate}[topsep=2pt, itemsep=2pt, leftmargin=*]
    \item Find the \textbf{general solution}.
    \item Find the \textbf{particular solution} given that $y(1)=4$.
    \item Find the \textbf{particular solution} given that $y(1)=3$.
  \end{enumerate}
  \item Answer the following parts corresponding to the \textbf{Second-Order Differential Equation}:
  {
  \begin{align}
    \frac{d^{2}y}{dt^{2}} = \cos(t)
  \end{align}
    }
  \begin{enumerate}[topsep=2pt, itemsep=2pt, leftmargin=*]
    \item Find the \textbf{particular solution} given that $y^{\prime}(0) = 0$ and $y(0) = 1$.
    \item Find the \textbf{particular solution} given that $y(1)=3$.
  \end{enumerate}
\end{enumerate}


\subsection{Graphical Interpretations }
\subsection{Initial Value Problems (continued)}



\newpage
% Lesson 03 - LINEAR EQUATIONS 
\section{LINEAR EQUATIONS}
\subsection{(2.3) - (General Overview) Linear Equations}
\subsection{Integrating Factor Method}
\subsection{Interpret and Understand the Structure and Behavior of Solutions and Their Domain}



\newpage
% Lesson 04 - EXACT EQUATIONS 
\section{EXACT EQUATIONS}
\subsection{(2.4) - (General Overview) Exact Equations}
\subsection{Classifying and Solving Potential Functions}
\subsection{Solving Equations by Integration and Applying Initial Conditions }



\newpage
% Lesson 05 - SOLUTIONS BY SUBSTITUTION
\section{SOLUTIONS BY SUBSTITUTION}
\subsection{(2.5) - (General Overview) Solutions By Substitutions }
\subsection{Homogeneous Equations}
\subsection{Bernoulli Equations}
\subsection{Other substitutions and methods}



\newpage
% Lesson 06 - LINEAR MODELS 
\section{LINEAR MODELS}
\subsection{(3.1) - (General Overview) Modeling with 1st Order Differential Equations}
\subsection{Exponential Growth and Decay}
\subsection{Newton's Law Of Cooling}
\subsection{Mixing Problems}
\subsection{Proportional Change Models}

\section{SOLUTIONS}


\newpage
\end{document}